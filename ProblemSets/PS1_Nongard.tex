\documentclass{article}
\usepackage[utf8]{inputenc}

\title{PS1}
\author{Alex Nongard}
\date{January 2018}

\begin{document}

\maketitle

\section{Introduction}

The world of economics is wide, but my interests lay mostly in energy and environmental pursuits. With a changing climate, growing population, and shifting humanity-wide priorities, I see the next few decades as crucial in managing substantial changes to the way we live. One of the biggest needs we must address is peoples' desire for clean and efficient energy. Outdated models of dirty, centralized, fossil-fueled power plants that are unreliable, dangerous, polluting, and un-nimble must go.

However, it will be difficult. Steep regulation and a lack of money behind clean energy makes fighting the establishment difficult. I intend to solve these problems, and work at the local level (in city government, if possible) to pilot my policy ideas. 

Outside of energy and environmental economics, I am interested in segregation studies. Residential racial segregation is not a problem that ended with the civil rights era - it's pervasive and damaging and persists in most major cities in America. The ways in which we categorize, measure, and create policy prescriptions for the issue is fascinating to me, and there's a large body of economic work to be done on it. 

I took this class because I believe it will give me practical skills to qualify me for work in these policy spheres. I will learn tools and applications that will make me a better contributor to policy solutions. My plan is to apply methods from this class to my solar site suitability project. I hope this course will teach me project management from beginning to end. 

After graduation in May (wow), I want to work for the City of Tulsa, the Federal Reserve, or some other public institution doing economic research, specifically on energy and the environment. 

\section{Equation}
a^2+b^2=c^2


\end{document}
