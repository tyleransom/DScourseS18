PS6Thatcher
rachel.e.thatcher-1
March 2018

Question 3
I didn�t really have to clean that data much. I took Shreya�s suggestion and used the iris data, because it is good for beginners and learning purposes. I tried using my data from last class, that way it would be a bit more �wild,� but I think my R updated or something because I was having trouble running code that I wrote two weeks ago and worked ?ne then but now is running into a lot of problems. So, a bit of a cop out on the data and data cleaning front.

Question 4
Again, I was pretty basic here. I tried some more complex plots than scatter, but some just didn�t work well with the data, and some just didn�t work...

Plot 1
So for the ?rst plot I kept it pretty basic, just making the basic scatter plot interconnecting all the data. This is useful to start o? because it shows how each of the variables are related to each other. So, for example, looking at species and petal length and width we can see that there is some separation based on which variation of the ?ower is being looked at. However, looking at species and sepal length and width there is much less of a di?erence between species. I also choose to make each species the color of the ?ower, so Setosa is a the deep blue/purple color, Versicolor is the more maroon/purple color, and Virginica is the light purple.

Plot 2
The second plot looked at how petal length and petal width are related. It appears fairly linear, with a longer petal making for a larger petal width. It also seems to be grouped by species for the most part, with Setosa being the smallest petals, Versicolor being in the middle, and Virginica being the largest.

Plot 3
The last plot shows the sepal length in relation to the sepal width. Unlike with petals, there is less of a linear relationship between the length of a sepal and its width. However, if you look at each species individually, there appears to be a bit more of a relationship, with the general trend being the longer the sepal the wider it is. Also looking at the groupings by species it seems that Setosa are fairly di?erent from the other two species, and that their sepal�s are wider and shorter while both Versicolor and Virginica both have longer and thinner sepals.