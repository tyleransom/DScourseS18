% Fonts/languages
\documentclass[12pt,english]{exam}
\IfFileExists{lmodern.sty}{\usepackage{lmodern}}{}
\usepackage[T1]{fontenc}
\usepackage[latin9]{inputenc}
\usepackage{babel}
\usepackage{mathpazo}
%\usepackage{mathptmx}

% Colors: see  http://www.math.umbc.edu/~rouben/beamer/quickstart-Z-H-25.html
\usepackage{color}
\usepackage[dvipsnames]{xcolor}
\definecolor{byublue}     {RGB}{0.  ,30. ,76. }
\definecolor{deepred}     {RGB}{190.,0.  ,0.  }
\definecolor{deeperred}   {RGB}{160.,0.  ,0.  }
\newcommand{\textblue}[1]{\textcolor{byublue}{#1}}
\newcommand{\textred}[1]{\textcolor{deeperred}{#1}}

% Layout
\usepackage{setspace} %singlespacing; onehalfspacing; doublespacing; setstretch{1.1}
\setstretch{1.2}
\usepackage[verbose,nomarginpar,margin=1in]{geometry} % Margins
\setlength{\headheight}{15pt} % Sufficent room for headers
\usepackage[bottom]{footmisc} % Forces footnotes on bottom

% Headers/Footers
\setlength{\headheight}{15pt}	
%\usepackage{fancyhdr}
%\pagestyle{fancy}
%\lhead{For-Profit Notes} \chead{} \rhead{\thepage}
%\lfoot{} \cfoot{} \rfoot{}

% Useful Packages
%\usepackage{bookmark} % For speedier bookmarks
\usepackage{amsthm}   % For detailed theorems
\usepackage{amssymb}  % For fancy math symbols
\usepackage{amsmath}  % For awesome equations/equation arrays
\usepackage{array}    % For tubular tables
\usepackage{longtable}% For long tables
\usepackage[flushleft]{threeparttable} % For three-part tables
\usepackage{multicol} % For multi-column cells
\usepackage{graphicx} % For shiny pictures
\usepackage{subfig}   % For sub-shiny pictures
\usepackage{enumerate}% For cusomtizable lists
\usepackage{pstricks,pst-node,pst-tree,pst-plot} % For trees
\usepackage{listings}
\lstset{basicstyle=\ttfamily\footnotesize,breaklines=true}

% Bib
\usepackage[authoryear]{natbib} % Bibliography
\usepackage{url}                % Allows urls in bib

% TOC
\setcounter{tocdepth}{4}

% Links
\usepackage{hyperref}    % Always add hyperref (almost) last
\hypersetup{colorlinks,breaklinks,citecolor=black,filecolor=black,linkcolor=byublue,urlcolor=blue,pdfstartview={FitH}}
\usepackage[all]{hypcap} % Links point to top of image, builds on hyperref
\usepackage{breakurl}    % Allows urls to wrap, including hyperref

\pagestyle{head}
\firstpageheader{\textbf{\class\ - \term}}{\textbf{\examnum}}{\textbf{Due: Apr. 17\\ beginning of class}}
\runningheader{\textbf{\class\ - \term}}{\textbf{\examnum}}{\textbf{Due: Apr. 17\\ beginning of class}}
\runningheadrule

\newcommand{\class}{Econ 5970}
\newcommand{\term}{Spring 2018}
\newcommand{\examdate}{Due: April 17, 2018}
% \newcommand{\timelimit}{30 Minutes}

\noprintanswers                         % Uncomment for no solutions version
\newcommand{\examnum}{Problem Set 11}           % Uncomment for no solutions version
% \printanswers                           % Uncomment for solutions version
% \newcommand{\examnum}{Problem Set 11 - Solutions} % Uncomment for solutions version

\begin{document}
This problem set will have you begin working on putting together your final project materials.

As with the previous problem sets, you will submit this problem set by pushing the document to \emph{your} (private) fork of the class repository. You will put this and all other problem sets in the path \texttt{/DScourseS18/ProblemSets/PS11/} and name the file \texttt{PS11\_LastName.*}. Your OSCER home directory and GitHub repository should be perfectly in sync, such that I should be able to find these materials by looking in either place. Your directory should contain at least three files:
\begin{itemize}
    \item \texttt{PS11\_LastName.bib}
    \item \texttt{PS11\_LastName.tex}
    \item \texttt{PS11\_LastName.pdf}
\end{itemize}
\begin{questions}
\question Type \texttt{git pull origin master} from your OSCER \texttt{DScourseS18} folder to make sure your OSCER folder is synchronized with your GitHub repository. 

\question Synchronize your fork with the class repository by doing a \texttt{git fetch upstream} and then merging the resulting branch. (\texttt{git merge upstream/master -m ``commit message''} followed by \texttt{git push origin master})

\question Begin compiling a Bib\TeX file (.bib extension) that will house the references for your final project writuep.

\question Turn in a rough draft of your final project submission. You should turn in the \LaTeX file as well as the corresponding PDF output. I'd like to see all sections of the final project attempted; however you do not have to have polished prose. (e.g. bullet lists and other summaries are acceptable)

\question Compile your .tex file, download the PDF and .tex file, and transfer it to your cloned repository on OSCER using your SFTP client of choice (or via \texttt{scp} from your laptop terminal). You may also copy and paste your .tex file from your browser directly into your terminal via \texttt{nano} if you prefer, but you will need to use SFTP or \texttt{scp} to transer the PDF.\footnote{If you want to try out something different, you can compile your .tex file on OSCER by typing \texttt{pdflatex myfile.tex} at the command prompt of the appropriate directory. This will create the PDF directly on OSCER, removing the requirement to use SFTP or \texttt{scp} to move the file over.}

\question You should turn in the following files: .tex, .pdf, and any additional scripts (e.g. .R, .py, or .jl) required to reproduce your work.  Make sure that these files each have the correct naming convention (see top of this problem set for directions) and are located in the correct directory (i.e. \texttt{\textasciitilde/DScourseS18/ProblemSets/PS11}).

\question Synchronize your local git repository (in your OSCER home directory) with your GitHub fork by using the commands in Problem Set 2 (i.e. \texttt{git add}, \texttt{git commit -m ''message''}, and \texttt{git push origin master}). Once you have done this, issue a \texttt{git pull} from the location of your other local git repository (e.g. on your personal computer). Verify that the PS11 files appear in the appropriate place in your other local repository.

\end{questions}
\end{document}
